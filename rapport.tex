\documentclass{article}
\usepackage{graphicx}
\graphicspath{ {./figures/} }

\title{TP MATLAB : Transformée de Fourier}
\date{9 Janvier 2021}
\author{Lucas LOISEAU \\ Alexandre SENOUCI \\ Maya BROYER \\ Lena BELIAZI}

\begin{document}
\maketitle
\section{Transformée de Fourier discrète}
Les signaux sont représentés sur l'intervalle $[-5,5]$. \\
On fixe le nombre d'échantillons à $N=32768$. \\
La période d'échantillonage vaut donc $T_e = 3,0518 * 10^{-4}$ et la fréquence d'échantillonage $f_e=3276,8$.11111
\subsection{Echantillonage et spectre de fonctions usuelles}
\subsubsection{Fonction cosinus}
On considère la fonction $x_1(t)=\cos(2\pi f_0 t)$ avec $f_0=4$, sa transformée de Fourier théorique est $X_1(f)=\frac{1}{2}[\delta(f-f_0)+\delta(f+f_0)]$, il s'agit d'un nombre réel.
\begin{figure}[h]
\includegraphics[scale=0.63]{fig_cos}
\centering
\end{figure}

\subsubsection{Fonction sinus}
On considère la fonction $x_2=\sin(2\pi f_0 t)$ avec $f_0=4$, sa transformée de Fourier théorique est $X_2(f)=\frac{1}{2j}[\delta(f-f_0)-\delta(f+f_0)]$, il s'agit d'un imaginaire pur.
\begin{figure}[h]
\includegraphics[scale=0.6]{fig_sin}
\centering
\end{figure}

\subsubsection{Fonction pic de Dirac}
On considère la fonction $x_3(t)=\delta(t-\Delta t)$, sa transformée de Fourier théorique est $X_3(f)=e^{-j2\pi f\Delta t}$. \\
Lorsque $\Delta t$ est nul, nous constatons bien que la transformée de Fourier du Dirac est constante à $1$ :
\begin{figure}[h]
\includegraphics[scale=0.6]{fig_dirac_0}
\centering
\end{figure} \\
Tandis qu'avec un décalage temporel $\Delta t$ non nul, nous obtenons les courbes suivantes :
\begin{figure}[h]
\includegraphics[scale=0.6]{fig_dirac_deltat}
\centering
\end{figure} \\
Ce qui correspond bien $Re(X_3(f))=\cos(2\pi f\Delta t)$ et $Im(X_3(f))=\sin(2\pi f\Delta t)$

\subsubsection{Exponentielle complexe}
On considère la fonction $x_4=e^{j2\pi f_0 t}$ qui a comme transformée de Fourier théorique $X_4=\delta(f-f_0)$.
\begin{figure}[h]
\includegraphics[scale=0.6]{fig_exp}
\centering
\end{figure} \\
La transformée de Fourier donne bien $\delta(f-f_0)$ (avec $f_0=4$ sur cette figure).

\subsubsection{Foncion rectangle}
\paragraph{Fonction rectangle apériodique}
Nous allons analyser la fonction rectangle dont la transformée de Fourier théorique est un sinus cardinal.
\begin{figure}[h]
\includegraphics[scale=0.6]{fig_rect}
\centering
\end{figure}
\paragraph{Fonction rectangle périodique}
Nous considérons cette fois une fonction rectangle rendue périodique.
\begin{figure}[h]
\includegraphics[scale=0.6]{fig_rect_per}
\centering
\end{figure} \\
La transformée de Fourier donne cette fois un sinus cardinal discrétisé.
\subsubsection{Courbe de Gauss}
La dernière fonction que nous allons analyser est la une courbe de Gauss dont l'équation est $x_6(t)=\exp(-\pi t^2)$
\begin{figure}[h]
\includegraphics[scale=0.6]{fig_gauss}
\centering
\end{figure}
\end{document}